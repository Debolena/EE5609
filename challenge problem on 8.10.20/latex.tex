\documentclass[journal,12pt,twocolumn]{IEEEtran}
%
\usepackage{setspace}
\usepackage{gensymb}
%\doublespacing
\singlespacing

%\usepackage{graphicx}
%\usepackage{amssymb}
%\usepackage{relsize}
\usepackage[cmex10]{amsmath}
%\usepackage{amsthm}
%\interdisplaylinepenalty=2500
%\savesymbol{iint}
%\usepackage{txfonts}
%\restoresymbol{TXF}{iint}
%\usepackage{wasysym}
\usepackage{amsthm}
%\usepackage{iithtlc}
\usepackage{mathrsfs}
\usepackage{txfonts}
\usepackage{stfloats}
\usepackage{bm}
\usepackage{cite}
\usepackage{cases}
\usepackage{subfig}
%\usepackage{xtab}
\usepackage{longtable}
\usepackage{multirow}
%\usepackage{algorithm}
%\usepackage{algpseudocode}
\usepackage{enumitem}
\usepackage{mathtools}
\usepackage{steinmetz}
\usepackage{tikz}
\usepackage{circuitikz}
\usepackage{verbatim}
\usepackage{tfrupee}
\usepackage[breaklinks=true]{hyperref}
%\usepackage{stmaryrd}
\usepackage{tkz-euclide} % loads  TikZ and tkz-base
%\usetkzobj{all}
\usetikzlibrary{calc,math}
\usepackage{listings}
    \usepackage{color}                                            %%
    \usepackage{array}                                            %%
    \usepackage{longtable}                                        %%
    \usepackage{calc}                                             %%
    \usepackage{multirow}                                         %%
    \usepackage{hhline}                                           %%
    \usepackage{ifthen}                                           %%
  %optionally (for landscape tables embedded in another document): %%
    \usepackage{lscape}     
\usepackage{multicol}
\usepackage{chngcntr}
%\usepackage{enumerate}

%\usepackage{wasysym}
%\newcounter{MYtempeqncnt}
\DeclareMathOperator*{\Res}{Res}
%\renewcommand{\baselinestretch}{2}
\renewcommand\thesection{\arabic{section}}
\renewcommand\thesubsection{\thesection.\arabic{subsection}}
\renewcommand\thesubsubsection{\thesubsection.\arabic{subsubsection}}

\renewcommand\thesectiondis{\arabic{section}}
\renewcommand\thesubsectiondis{\thesectiondis.\arabic{subsection}}
\renewcommand\thesubsubsectiondis{\thesubsectiondis.\arabic{subsubsection}}

% correct bad hyphenation here
\hyphenation{op-tical net-works semi-conduc-tor}
\def\inputGnumericTable{}                                 %%

\lstset{
%language=C,
frame=single, 
breaklines=true,
columns=fullflexible
}
%\lstset{
%language=tex,
%frame=single, 
%breaklines=true
%}

\begin{document}
%


\newtheorem{theorem}{Theorem}[section]
\newtheorem{problem}{Problem}
\newtheorem{proposition}{Proposition}[section]
\newtheorem{lemma}{Lemma}[section]
\newtheorem{corollary}[theorem]{Corollary}
\newtheorem{example}{Example}[section]
\newtheorem{definition}[problem]{Definition}
%\newtheorem{thm}{Theorem}[section] 
%\newtheorem{defn}[thm]{Definition}
%\newtheorem{algorithm}{Algorithm}[section]
%\newtheorem{cor}{Corollary}
\newcommand{\BEQA}{\begin{eqnarray}}
\newcommand{\EEQA}{\end{eqnarray}}
\newcommand{\define}{\stackrel{\triangle}{=}}
\bibliographystyle{IEEEtran}
%\bibliographystyle{ieeetr}
\providecommand{\mbf}{\mathbf}
\providecommand{\pr}[1]{\ensuremath{\Pr\left(#1\right)}}
\providecommand{\qfunc}[1]{\ensuremath{Q\left(#1\right)}}
\providecommand{\sbrak}[1]{\ensuremath{{}\left[#1\right]}}
\providecommand{\lsbrak}[1]{\ensuremath{{}\left[#1\right.}}
\providecommand{\rsbrak}[1]{\ensuremath{{}\left.#1\right]}}
\providecommand{\brak}[1]{\ensuremath{\left(#1\right)}}
\providecommand{\lbrak}[1]{\ensuremath{\left(#1\right.}}
\providecommand{\rbrak}[1]{\ensuremath{\left.#1\right)}}
\providecommand{\cbrak}[1]{\ensuremath{\left\{#1\right\}}}
\providecommand{\lcbrak}[1]{\ensuremath{\left\{#1\right.}}
\providecommand{\rcbrak}[1]{\ensuremath{\left.#1\right\}}}
\theoremstyle{remark}
\newtheorem{rem}{Remark}
\newcommand{\sgn}{\mathop{\mathrm{sgn}}}
\providecommand{\abs}[1]{\left\vert#1\right\vert}
\providecommand{\res}[1]{\Res\displaylimits_{#1}} 
\providecommand{\norm}[1]{\left\lVert#1\right\rVert}
%\providecommand{\norm}[1]{\lVert#1\rVert}
\providecommand{\mtx}[1]{\mathbf{#1}}
\providecommand{\mean}[1]{E\left[ #1 \right]}
\providecommand{\fourier}{\overset{\mathcal{F}}{ \rightleftharpoons}}
%\providecommand{\hilbert}{\overset{\mathcal{H}}{ \rightleftharpoons}}
\providecommand{\system}{\overset{\mathcal{H}}{ \longleftrightarrow}}
	%\newcommand{\solution}[2]{\textbf{Solution:}{#1}}
\newcommand{\solution}{\noindent \textbf{Solution: }}
\newcommand{\cosec}{\,\text{cosec}\,}
\providecommand{\dec}[2]{\ensuremath{\overset{#1}{\underset{#2}{\gtrless}}}}
\newcommand{\myvec}[1]{\ensuremath{\begin{pmatrix}#1\end{pmatrix}}}
\newcommand{\mydet}[1]{\ensuremath{\begin{vmatrix}#1\end{vmatrix}}}
%\numberwithin{equation}{section}
\numberwithin{equation}{subsection}
%\numberwithin{problem}{section}
%\numberwithin{definition}{section}
\makeatletter
\@addtoreset{figure}{problem}
\makeatother
\let\StandardTheFigure\thefigure
\let\vec\mathbf
%\renewcommand{\thefigure}{\theproblem.\arabic{figure}}
\renewcommand{\thefigure}{\theproblem}
%\setlist[enumerate,1]{before=\renewcommand\theequation{\theenumi.\arabic{equation}}
%\counterwithin{equation}{enumi}
%\renewcommand{\theequation}{\arabic{subsection}.\arabic{equation}}
\def\putbox#1#2#3{\makebox[0in][l]{\makebox[#1][l]{}\raisebox{\baselineskip}[0in][0in]{\raisebox{#2}[0in][0in]{#3}}}}
     \def\rightbox#1{\makebox[0in][r]{#1}}
     \def\centbox#1{\makebox[0in]{#1}}
     \def\topbox#1{\raisebox{-\baselineskip}[0in][0in]{#1}}
     \def\midbox#1{\raisebox{-0.5\baselineskip}[0in][0in]{#1}}
\vspace{3cm}
\title{Challenge Question: Matrix Theory}
\author{Debolena Basak\\PhD Artificial Intelligence\\ Roll No.: AI20RESCH11003}

\maketitle
\newpage
%\tableofcontents
\bigskip
\renewcommand{\thefigure}{\theenumi}
\renewcommand{\thetable}{\theenumi}

\begin{abstract}
This is a problem of a point, circle and tangent.
\end{abstract}

Download all the latex-tikz codes from 
%
\begin{lstlisting}
https://github.com/Debolena/EE5609/tree/master/challenge%20problem%20on%208.10.20
\end{lstlisting}
%
\section{Problem}
Given a point $\vec P$ outside a circle whose equation is known.  Find the equation(s) of the tangent(s) and the the distance from $\vec P$ to the point(s) of contact.

%\input{./chapters/problem.tex}

%In right triangle ABC, right angled at C, M is
%the mid-point of hypotenuse AB. C is joined to
%M and produced to a point D such that DM =
%CM. Point D is joined to point B. Show that
%
%\begin{enumerate}[label = (\alph*)]
%\item $\triangle  AMC  \cong   \triangle  BMD $
%\item $\triangle DBC $ is a right angle.
%\item $\triangle  DBC  \cong  \triangle  ABC $
%\item $CM = \frac{1}{2} AB$
%\end{enumerate}
%\section{Construction}
%\input{./chapters/constr.tex}
\section{Solution}
The equation of the circle can be expressed as 
\begin{align}
     \vec x^T\vec x + 2 \vec u^T\vec x+f=0
\end{align}
Let point 
\begin{align}
    \vec P= \myvec{p_1\\p_2}
\end{align}
We know radius of circle and centre is given by 
\begin{align}
     \sqrt{{\norm {\vec u}}^2-f}\\
     and \quad \vec c= -\vec u
\end{align}
,respectively.
Let the tangents from point $\vec P$ touch the circle at $\vec q$ and $\vec r$. 
At point $\vec q$,
the normal vector is:
\begin{align}
    &\vec V \vec q +\vec u \\
    &= \vec q -\vec c, \quad\because \vec V= \vec I\\
    & = \vec n_1, say.
\end{align}
Similarly, at point $\vec r$, the normal vector is:
\begin{align}
    &\vec r - \vec c \\
    & = \vec n_2, say
\end{align}
Then, the equation of the tangents are:
\begin{align}
    &\label{tangent1}\vec n_1^T \brak{\vec x - \vec q}=0\\
    &\label{tangent2}\vec n_2^T \brak{\vec x - \vec r}=0
\end{align}
We also have points of contact as,
\begin{align}
    &\vec q= k_1\vec n_1-\vec u\\
    &\implies \label{contact1}\vec q= k_1\vec n_1+\vec c\\
    & \quad and\\
    & \vec r= k_2\vec n_2-\vec u\\
    &\implies \label{contact2}\vec r= k_2\vec n_2+\vec c
\end{align}
where, 
\begin{align}
    &\kappa_i = \pm \sqrt{\frac{\vec u^T \vec V^{-1} \vec u -f}{\vec n^T \vec V^{-1} \vec n}}\\
    &= \pm \sqrt{\frac{{\norm {\vec u}}^2 -f}{\vec {n_i}^T \vec {n_i}}}\\
    &= \pm \frac{radius}{\norm {\vec {n_i}}}
\end{align}
Using \ref{tangent1}, \ref{tangent2}, \ref{contact1} and \ref{contact2}, we can solve for the points of contact $\vec q$ and $\vec r$.
Therefore, distance from $\vec P$ to the points of contact are:
\begin{align}
    &\norm{\vec P -\vec q},\\
    & \norm{\vec P-\vec r}
\end{align}
%\vspace{5mm} 

%\input{./chapters/solution.tex}
%\subsection{Sol.a)}
%\input{./chapters/sol_a.tex}
%\subsection{Sol.b)}
%\input{./chapters/sol_b.tex}
%\subsection{Sol.c)}
%\input{./chapters/sol_c.tex}
%\subsection{Sol.d)}
%\input{chapters/sol_d.tex}
\end{document}
