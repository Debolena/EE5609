\documentclass[journal,12pt,twocolumn]{IEEEtran}
%
\usepackage{setspace}
\usepackage{gensymb}
%\doublespacing
\singlespacing

%\usepackage{graphicx}
%\usepackage{amssymb}
%\usepackage{relsize}
\usepackage[cmex10]{amsmath}
%\usepackage{amsthm}
%\interdisplaylinepenalty=2500
%\savesymbol{iint}
%\usepackage{txfonts}
%\restoresymbol{TXF}{iint}
%\usepackage{wasysym}
\usepackage{amsthm}
%\usepackage{iithtlc}
\usepackage{mathrsfs}
\usepackage{txfonts}
\usepackage{stfloats}
\usepackage{bm}
\usepackage{cite}
\usepackage{cases}
\usepackage{subfig}
%\usepackage{xtab}
\usepackage{longtable}
\usepackage{multirow}
%\usepackage{algorithm}
%\usepackage{algpseudocode}
\usepackage[utf8]{inputenc}
\usepackage{tikz}
\usetikzlibrary{positioning}
\usepackage{enumitem}
\usepackage{mathtools}
\usepackage{steinmetz}
\usepackage{tikz}
\usepackage{circuitikz}
\usepackage{verbatim}
\usepackage{tfrupee}
\usepackage[breaklinks=true]{hyperref}
%\usepackage{stmaryrd}
\usepackage{tkz-euclide} % loads  TikZ and tkz-base
%\usetkzobj{all}
\usetikzlibrary{calc,math}
\usepackage{listings}
    \usepackage{color}                                            %%
    \usepackage{array}                                            %%
    \usepackage{longtable}                                        %%
    \usepackage{calc}                                             %%
    \usepackage{multirow}                                         %%
    \usepackage{hhline}                                           %%
    \usepackage{ifthen}                                           %%
  %optionally (for landscape tables embedded in another document): %%
    \usepackage{lscape}     
\usepackage{multicol}
\usepackage{chngcntr}
%\usepackage{enumerate}

%\usepackage{wasysym}
%\newcounter{MYtempeqncnt}
\DeclareMathOperator*{\Res}{Res}
%\renewcommand{\baselinestretch}{2}
\renewcommand\thesection{\arabic{section}}
\renewcommand\thesubsection{\thesection.\arabic{subsection}}
\renewcommand\thesubsubsection{\thesubsection.\arabic{subsubsection}}

\renewcommand\thesectiondis{\arabic{section}}
\renewcommand\thesubsectiondis{\thesectiondis.\arabic{subsection}}
\renewcommand\thesubsubsectiondis{\thesubsectiondis.\arabic{subsubsection}}

% correct bad hyphenation here
\hyphenation{op-tical net-works semi-conduc-tor}
\def\inputGnumericTable{}                                 %%

\lstset{
%language=C,
frame=single, 
breaklines=true,
columns=fullflexible
}
%\lstset{
%language=tex,
%frame=single, 
%breaklines=true
%}

\begin{document}
%


\newtheorem{theorem}{Theorem}[section]
\newtheorem{problem}{Problem}
\newtheorem{proposition}{Proposition}[section]
\newtheorem{lemma}{Lemma}[section]
\newtheorem{corollary}[theorem]{Corollary}
\newtheorem{example}{Example}[section]
\newtheorem{definition}[problem]{Definition}
%\newtheorem{thm}{Theorem}[section] 
%\newtheorem{defn}[thm]{Definition}
%\newtheorem{algorithm}{Algorithm}[section]
%\newtheorem{cor}{Corollary}
\newcommand{\BEQA}{\begin{eqnarray}}
\newcommand{\EEQA}{\end{eqnarray}}
\newcommand{\define}{\stackrel{\triangle}{=}}
\bibliographystyle{IEEEtran}
%\bibliographystyle{ieeetr}
\providecommand{\mbf}{\mathbf}
\providecommand{\pr}[1]{\ensuremath{\Pr\left(#1\right)}}
\providecommand{\qfunc}[1]{\ensuremath{Q\left(#1\right)}}
\providecommand{\sbrak}[1]{\ensuremath{{}\left[#1\right]}}
\providecommand{\lsbrak}[1]{\ensuremath{{}\left[#1\right.}}
\providecommand{\rsbrak}[1]{\ensuremath{{}\left.#1\right]}}
\providecommand{\brak}[1]{\ensuremath{\left(#1\right)}}
\providecommand{\lbrak}[1]{\ensuremath{\left(#1\right.}}
\providecommand{\rbrak}[1]{\ensuremath{\left.#1\right)}}
\providecommand{\cbrak}[1]{\ensuremath{\left\{#1\right\}}}
\providecommand{\lcbrak}[1]{\ensuremath{\left\{#1\right.}}
\providecommand{\rcbrak}[1]{\ensuremath{\left.#1\right\}}}
\theoremstyle{remark}
\newtheorem{rem}{Remark}
\newcommand{\sgn}{\mathop{\mathrm{sgn}}}
\providecommand{\abs}[1]{\left\vert#1\right\vert}
\providecommand{\res}[1]{\Res\displaylimits_{#1}} 
\providecommand{\norm}[1]{\left\lVert#1\right\rVert}
%\providecommand{\norm}[1]{\lVert#1\rVert}
\providecommand{\mtx}[1]{\mathbf{#1}}
\providecommand{\mean}[1]{E\left[ #1 \right]}
\providecommand{\fourier}{\overset{\mathcal{F}}{ \rightleftharpoons}}
%\providecommand{\hilbert}{\overset{\mathcal{H}}{ \rightleftharpoons}}
\providecommand{\system}{\overset{\mathcal{H}}{ \longleftrightarrow}}
	%\newcommand{\solution}[2]{\textbf{Solution:}{#1}}
\newcommand{\solution}{\noindent \textbf{Solution: }}
\newcommand{\cosec}{\,\text{cosec}\,}
\providecommand{\dec}[2]{\ensuremath{\overset{#1}{\underset{#2}{\gtrless}}}}
\newcommand{\myvec}[1]{\ensuremath{\begin{pmatrix}#1\end{pmatrix}}}
\newcommand{\mydet}[1]{\ensuremath{\begin{vmatrix}#1\end{vmatrix}}}
%\numberwithin{equation}{section}
\numberwithin{equation}{subsection}
%\numberwithin{problem}{section}
%\numberwithin{definition}{section}
\makeatletter
\@addtoreset{figure}{problem}
\makeatother
\let\StandardTheFigure\thefigure
\let\vec\mathbf
%\renewcommand{\thefigure}{\theproblem.\arabic{figure}}
\renewcommand{\thefigure}{\theproblem}
%\setlist[enumerate,1]{before=\renewcommand\theequation{\theenumi.\arabic{equation}}
%\counterwithin{equation}{enumi}
%\renewcommand{\theequation}{\arabic{subsection}.\arabic{equation}}
\def\putbox#1#2#3{\makebox[0in][l]{\makebox[#1][l]{}\raisebox{\baselineskip}[0in][0in]{\raisebox{#2}[0in][0in]{#3}}}}
     \def\rightbox#1{\makebox[0in][r]{#1}}
     \def\centbox#1{\makebox[0in]{#1}}
     \def\topbox#1{\raisebox{-\baselineskip}[0in][0in]{#1}}
     \def\midbox#1{\raisebox{-0.5\baselineskip}[0in][0in]{#1}}
\vspace{3cm}
\title{Matrix Theory: Assignment 6}
\author{Debolena Basak\\ Roll No.: AI20RESCH11003\\ PhD Artificial Intelligence}

\maketitle
\newpage
%\tableofcontents
\bigskip
\renewcommand{\thefigure}{\theenumi}
\renewcommand{\thetable}{\theenumi}


\begin{abstract}
This document is to find the QR decomposition of $\vec V$ and vertex of a parabola using SVD, then verifying it using Least Squares.
\end{abstract}

%
Download all latex-tikz codes from 
%
\begin{lstlisting}
https://github.com/Debolena/EE5609/tree/master/Assignment_6
\end{lstlisting}
%
\section{Problem}
From Assignment 5,\\
1. Find the QR decomposition of V.\\
2. Obtain $\vec{c}$ using SVD by using $\eta/2 $ instead of $\eta$ and verify your solution using least squares.

\section{Solution}
\subsection{$\vec{QR}$ decompososition of $\vec V$}
We have, 
\begin{align}
    \vec V = \myvec{1 & -4\\ -4 & 16} \label{eq:1}
\end{align}

where, $\vec{V}$ can be written as,
\begin{align}
    \vec{V} = \myvec{\vec{a} & \vec{b}} \label{eq:2}
\end{align}
where $\vec{a}$ and $\vec{b}$ and  are column vectors.
Here, 
\begin{align}
    \vec{a} = \myvec{1 \\ -4} \label{eq:3}
\end{align}
\begin{align}
    \vec{b} = \myvec{-4 \\ 16} \label{eq:4}
\end{align}
The QR decomposition of a matrix $\vec{A}$ is given by, 
\begin{align}
    \vec{A} = \vec{QR}
\end{align}
where $\vec{R}$ is a upper triangular matrix and $\vec{Q}$ is such that, 
\begin{align}
    \vec{Q^TQ} = \vec{I}
\end{align}
where,
\begin{align}
    \vec{Q} = \myvec{\vec{q_1} & \vec{q_2}} \\[1em]
    \vec{R} = \myvec{r_1 & r_2 \\ 0 & r_3}    
\end{align}
The values in $\vec{R}$ and $\vec{q_1}$, $\vec{q_2}$  are given by,
\begin{align}
    r_1 = \norm{\vec{a}}\\[1em]
    \vec{q_1} = \frac{\vec{a}}{r_1}\\[1em]
    r_2 = \frac{\vec{q_1^Tb}}{\norm{\vec{q_1}}^{2}}\\[1em]
    \vec{q_2} = \frac{\vec{b} - r_2\vec{q_1}}{\norm{\vec{b}-r_2\vec{q_1}}}\\[1em]
    r_3 = \vec{q_2^Tb}
\end{align}
Using \eqref{eq:3}, \eqref{eq:4} and the above formulas, 
\begin{align}
    r_1= \sqrt{17}\\[1em]
    \vec{q_1}= \frac{1}{\sqrt{17}} \myvec{1\\-4} = \myvec{\frac{1}{\sqrt{17}}\\ -\frac{4}{\sqrt{17}}}\\[1em]
    \norm {\vec{q_1}} = 1\\[1em]
    r_2 = \myvec{\frac{1}{\sqrt{17}} & -\frac{4}{\sqrt{17}}} \myvec{-4\\16} = -\frac{68}{\sqrt{17}}\\[1em]
    \vec{q_2}= \myvec{0\\0}\\[1em]
    r_3= 0\\
\end{align}
Hence, 
\begin{align}
    \vec{QR} &= \myvec{\vec q_1 & \vec q_2} \myvec{r_1 & r_2\\ 0 & r_3}\\[1em]
    &= \myvec{\frac{1}{\sqrt{17}} & 0 \\ -\frac{4}{\sqrt{17}} & 0} \myvec{\sqrt{17} & -\frac{68}{\sqrt{17}}\\0 & 0}\\[1em]
    &= \myvec{1 & -4\\ -4 & 16} \label{eq:QR}
\end{align}
Clearly, \eqref{eq:QR} and \eqref{eq:1} are equal. Hence, the $\vec{QR}$ decomposition holds.
\subsection{Finding vertex using SVD}
\begin{align}
    &\vec{V} = \vec V^T= \myvec{1&-4\\-4&16}\label{eq: matrix V}\\ 
    &\vec{u} = \myvec{0\\-\frac{51}{2}}\label{eq: matrix U} \\ 
    &f = 0 \label{eq: f}\\
    &\vec{P}=\myvec{\vec{p_1}&\vec{p_2}}=\frac{1}{\sqrt{17}}\myvec{-4&-1\\ -1 & 4}\label{eq: matrix P}\\
    &\eta= \vec u^T \vec p_1 = \frac{51}{2\sqrt{17}}\label{eq: eta value}
\end{align}
The equation of perpendicular line passing through focus and intersecting parabola at vertex $\vec c$ is given as
\begin{align}
    \myvec{\vec{u^T}+\frac{\eta}{2}\vec{p_1^T} \\ \vec{V}}\vec{c}=\myvec{-f \\ \frac{\eta}{2}\vec{p_1}-\vec{u}} 
\end{align}
Using \eqref{eq: matrix U}, \eqref{eq: matrix P}, \eqref{eq: eta value}, \eqref{eq: matrix V} and \eqref{eq: f},
\begin{align}
    &\myvec{-3 & -\frac{105}{4}\\1 & -4\\-4 & 16}\vec c = \myvec{0\\-3\\\frac{99}{4}}\\
    &\implies \vec M \vec c= \vec b\label{eq: Mc=b}
\end{align}
where, 
\begin{align}
    &\vec M = \myvec{-3 & -\frac{105}{4}\\1 & -4\\-4 & 16}\\
    &\vec b = \myvec{0\\-3\\\frac{99}{4}}
\end{align}
To solve \eqref{eq: Mc=b}, we perform Singular Value Decomposition on $\vec{M}$ given as 
\begin{align}
	\vec{M = USV^T }\label{eq: SVD}
\end{align}
Putting this value of $\vec{M}$ in \eqref{eq: Mc=b}, we get
\begin{align}
	&\vec{USV^T}\vec{c} = \vec{b} \\
\implies& \vec{c} = \vec{VS_+U^T}\vec{b}\label{eq: c}
\end{align}
where, $\vec{S_+}$ is Moore-Penrose pseudo-inverse of $\vec{S}$. Columns of $\vec{U}$ are eigen-vectors of $\vec{MM^T}$, columns of $\vec{V}$ are eigen-vectors of $\vec{M^TM}$ and $\vec{S}$ is diagonal matrix of singular value of eigenvalues of $\vec{M^TM}$.
\begin{align}
    \vec {MM^T} &= \myvec{-3 & \frac{-105}{4}\\1 & -4\\ -4 & 16} \myvec {-3 & 1 & -4\\ \frac{-105}{4} & -4 & 16}\\[1em]
    & = \myvec{698.0625 & 102 & -408\\ 102 & 17 & -68\\-408 & -68 & 272}\\[1em]
    \vec{M^TM}=& \myvec{-3 & 1 & -4\\-\frac{105}{4} & -4 & 16} \myvec{-3& -\frac{105}{4}\\1 & -4\\-4 & 16}\\[1em]
    &= \myvec{26 & 10.75\\ 10.75 & 961.0625}
\end{align}
Eigen values of $\vec{M^TM}$ can be found out as
\begin{align}
	 &\abs{\vec{M^TM-\lambda I}} = 0\\[1em]
	 \implies & \mydet{26- \lambda & 10.75\\ 10.75 & 961.0625 -\lambda}=0
\end{align}	
Solving this we get the eigen values of $\vec{M^TM}$ as, 
\begin{align}
    \lambda_1 = 961.1861\\
    \lambda_2=  25.8764
\end{align}
The corresponding eigen vectors are:
\begin{align}
    \vec v_1 = \myvec{0.0115\\ 1}\\
    \vec v_2 = \myvec{-86.994\\ 1}
\end{align}
Normalizing the values,
\begin{align}
    \vec v_1 = \myvec{0.0115\\0.9999}\\
    \vec v_2 = \myvec{-0.9999\\0.0115}
\end{align}
Hence, 
\begin{align}
    \vec V=& \myvec{\vec v_1 & \vec v_2}\\
    &=\myvec{0.0115 & -0.9999\\0.9999 & 0.0115}
\end{align}
Eigen values of $\vec{MM^T}$ can be found by solving:
\begin{align}
     &\abs{\vec{MM^T}-\lambda \vec {I}} = 0\\[1em]
	 \implies & \mydet{698.0625- \lambda & 102 & -408\\ 102 & 17- \lambda & -68\\ -408 & -68 & 272-\lambda}=0
\end{align}
Solving this, we get the eigen values of $\vec{MM^T}$ as:
\begin{align}
    \lambda_3= 961.1861\\
    \lambda_4= 25.8764\\
    \lambda_5= 0
\end{align}
The corresponding eigen vectors after normalizing are:
\begin{align}
    \vec u_1 = \myvec{-0.8477\\-0.1286\\0.5146}\\
    \vec u_2= \myvec{0.5304\\-0.2056\\0.8224}\\
    \vec u_3= \myvec{0 \\0.9701\\ 0.2425}\\
    \therefore \vec U= \myvec{-0.8477& 0.5304 & 0\\
   -0.1286 & -0.2056 & 0.9701\\
    0.5146 & 0.8224 & 0.2425}
\end{align}
After computing the singular values from the eigen values, \begin{align}
    \vec S &= \myvec{\sqrt{\lambda _1} & 0\\ 0 & \sqrt{\lambda_2}\\ 0 & 0}\\
    &= \myvec {31.0030 & 0 \\ 0 & 5.0869 \\0 & 0}
\end{align}
Therefore we get the SVD of $\vec M $ as:
\begin{multline}
    \vec M = \myvec{-0.8477& 0.5304 & 0\\
   -0.1286 & -0.2056 & 0.9701\\
    0.5146 & 0.8224 & 0.2425} \myvec{31.0030 & 0 \\ 0 & 5.0869 \\0 & 0}\\ \myvec {0.0115 & -0.9999\\0.9999 & 0.0115}^T
\end{multline}
\begin{align}
     = \myvec{-3 & -26.2500\\1 &  -4\\ -4 &  16}
\end{align}
Moore- penrose pseudo inverse of $\vec S $ is:
\begin{align}
    \vec S_+ = \myvec{0.0323 &0 &0\\ 0 & 0.1966 & 0}
\end{align}
Putting the values in \eqref{eq: c},
\begin{align}
    \vec {U^Tb}= \myvec{13.1213\\  20.9721\\3.0923}\\
    \vec {S_+ U^T b}= \myvec{0.4238\\  4.1231}\\
    \vec {c}= \vec{S_+ U^T b}= \myvec{-4.1180\\0.4712}\label{eq: svd sol c}
\end{align}
Verifying the above solution using least squares:
\begin{align}
    &\vec{M^T M c} =\vec{ M^T b}\label{eq: LS}\\
    \implies & \vec {M^T Mc}= \myvec{-102\\408}\\
    \implies & \myvec{26 & \frac{43}{4}\\\frac{43}{4} & \frac{15377}{16}}\vec c = \myvec{-102\\408} 
\end{align}
Forming the augmented matrix and row-reducing it:
\begin{align}
    \myvec{26 & \frac{43}{4} & -102\\[1em]\frac{43}{4} & \frac{15377}{16} & 408}\\[1em]
	\xleftrightarrow[]{R_2\leftarrow R_1 -\frac{4}{43} 26 R_2} 
	\myvec{26& \frac{43}{4} & -102\\[1em] 0 & -\frac{397953}{172} & -\frac{46818}{43}}\\[1em]
	\xleftrightarrow[]{R_2\leftarrow -\frac{172}{397953} R_2} 
	\myvec{26 & \frac{43}{4} & -102\\[1em]0 & 1 & \frac{8}{17}}\\[1em]
	\xleftrightarrow[]{R_1\leftarrow \frac{1}{26} R_1} 
	\myvec{1 & \frac{43}{104} & -\frac{51}{13}\\[1em] 0 & 1 & \frac{8}{17}}\\[1em]
	\xleftrightarrow[]{R_1\leftarrow R_1 -\frac{43}{104}R_2}
	\myvec{1 & 0 & -\frac{70}{17}\\[1em] 0 & 1 & \frac{8}{17}} \label{eq: sol}
\end{align}
From \eqref{eq: sol}, 
\begin{align}
    \vec c & = \myvec{ -\frac{70}{17}\\ \frac{8}{17}}\\
     & = \myvec{-4.1176 \\ 0.4706}\label{eq:LS sol c}
\end{align}
Comparing \eqref{eq: svd sol c} and \eqref{eq:LS sol c}, it can be said that the solution of $\vec c$ is verified.
\end{document}

