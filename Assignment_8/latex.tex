\documentclass[journal,12pt,twocolumn]{IEEEtran}
%
\usepackage{setspace}
\usepackage{gensymb}
%\doublespacing
\singlespacing

%\usepackage{graphicx}
%\usepackage{amssymb}
%\usepackage{relsize}
\usepackage[cmex10]{amsmath}
%\usepackage{amsthm}
%\interdisplaylinepenalty=2500
%\savesymbol{iint}
%\usepackage{txfonts}
%\restoresymbol{TXF}{iint}
%\usepackage{wasysym}
\usepackage{amsthm}
%\usepackage{iithtlc}
\usepackage{mathrsfs}
\usepackage{txfonts}
\usepackage{stfloats}
\usepackage{bm}
\usepackage{cite}
\usepackage{cases}
\usepackage{subfig}
%\usepackage{xtab}
\usepackage{longtable}
\usepackage{multirow}
%\usepackage{algorithm}
%\usepackage{algpseudocode}
\usepackage[utf8]{inputenc}
\usepackage{tikz}
\usetikzlibrary{positioning}
\usepackage{enumitem}
\usepackage{mathtools}
\usepackage{steinmetz}
\usepackage{tikz}
\usepackage{circuitikz}
\usepackage{verbatim}
\usepackage{tfrupee}
\usepackage[breaklinks=true]{hyperref}
%\usepackage{stmaryrd}
\usepackage{tkz-euclide} % loads  TikZ and tkz-base
%\usetkzobj{all}
\usetikzlibrary{calc,math}
\usepackage{listings}
    \usepackage{color}                                            %%
    \usepackage{array}                                            %%
    \usepackage{longtable}                                        %%
    \usepackage{calc}                                             %%
    \usepackage{multirow}                                         %%
    \usepackage{hhline}                                           %%
    \usepackage{ifthen}                                           %%
  %optionally (for landscape tables embedded in another document): %%
    \usepackage{lscape}     
\usepackage{multicol}
\usepackage{chngcntr}
%\usepackage{enumerate}

%\usepackage{wasysym}
%\newcounter{MYtempeqncnt}
\DeclareMathOperator*{\Res}{Res}
%\renewcommand{\baselinestretch}{2}
\renewcommand\thesection{\arabic{section}}
\renewcommand\thesubsection{\thesection.\arabic{subsection}}
\renewcommand\thesubsubsection{\thesubsection.\arabic{subsubsection}}

\renewcommand\thesectiondis{\arabic{section}}
\renewcommand\thesubsectiondis{\thesectiondis.\arabic{subsection}}
\renewcommand\thesubsubsectiondis{\thesubsectiondis.\arabic{subsubsection}}

% correct bad hyphenation here
\hyphenation{op-tical net-works semi-conduc-tor}
\def\inputGnumericTable{}                                 %%

\lstset{
%language=C,
frame=single, 
breaklines=true,
columns=fullflexible
}
%\lstset{
%language=tex,
%frame=single, 
%breaklines=true
%}

\begin{document}
%


\newtheorem{theorem}{Theorem}[section]
\newtheorem{problem}{Problem}
\newtheorem{proposition}{Proposition}[section]
\newtheorem{lemma}{Lemma}[section]
\newtheorem{corollary}[theorem]{Corollary}
\newtheorem{example}{Example}[section]
\newtheorem{definition}[problem]{Definition}
%\newtheorem{thm}{Theorem}[section] 
%\newtheorem{defn}[thm]{Definition}
%\newtheorem{algorithm}{Algorithm}[section]
%\newtheorem{cor}{Corollary}
\newcommand{\BEQA}{\begin{eqnarray}}
\newcommand{\EEQA}{\end{eqnarray}}
\newcommand{\define}{\stackrel{\triangle}{=}}
\bibliographystyle{IEEEtran}
%\bibliographystyle{ieeetr}
\providecommand{\mbf}{\mathbf}
\providecommand{\pr}[1]{\ensuremath{\Pr\left(#1\right)}}
\providecommand{\qfunc}[1]{\ensuremath{Q\left(#1\right)}}
\providecommand{\sbrak}[1]{\ensuremath{{}\left[#1\right]}}
\providecommand{\lsbrak}[1]{\ensuremath{{}\left[#1\right.}}
\providecommand{\rsbrak}[1]{\ensuremath{{}\left.#1\right]}}
\providecommand{\brak}[1]{\ensuremath{\left(#1\right)}}
\providecommand{\lbrak}[1]{\ensuremath{\left(#1\right.}}
\providecommand{\rbrak}[1]{\ensuremath{\left.#1\right)}}
\providecommand{\cbrak}[1]{\ensuremath{\left\{#1\right\}}}
\providecommand{\lcbrak}[1]{\ensuremath{\left\{#1\right.}}
\providecommand{\rcbrak}[1]{\ensuremath{\left.#1\right\}}}
\theoremstyle{remark}
\newtheorem{rem}{Remark}
\newcommand{\sgn}{\mathop{\mathrm{sgn}}}
\providecommand{\abs}[1]{\left\vert#1\right\vert}
\providecommand{\res}[1]{\Res\displaylimits_{#1}} 
\providecommand{\norm}[1]{\left\lVert#1\right\rVert}
%\providecommand{\norm}[1]{\lVert#1\rVert}
\providecommand{\mtx}[1]{\mathbf{#1}}
\providecommand{\mean}[1]{E\left[ #1 \right]}
\providecommand{\fourier}{\overset{\mathcal{F}}{ \rightleftharpoons}}
%\providecommand{\hilbert}{\overset{\mathcal{H}}{ \rightleftharpoons}}
\providecommand{\system}{\overset{\mathcal{H}}{ \longleftrightarrow}}
	%\newcommand{\solution}[2]{\textbf{Solution:}{#1}}
\newcommand{\solution}{\noindent \textbf{Solution: }}
\newcommand{\cosec}{\,\text{cosec}\,}
\providecommand{\dec}[2]{\ensuremath{\overset{#1}{\underset{#2}{\gtrless}}}}
\newcommand{\myvec}[1]{\ensuremath{\begin{pmatrix}#1\end{pmatrix}}}
\newcommand{\mydet}[1]{\ensuremath{\begin{vmatrix}#1\end{vmatrix}}}
%\numberwithin{equation}{section}
\numberwithin{equation}{subsection}
%\numberwithin{problem}{section}
%\numberwithin{definition}{section}
\makeatletter
\@addtoreset{figure}{problem}
\makeatother
\let\StandardTheFigure\thefigure
\let\vec\mathbf
%\renewcommand{\thefigure}{\theproblem.\arabic{figure}}
\renewcommand{\thefigure}{\theproblem}
%\setlist[enumerate,1]{before=\renewcommand\theequation{\theenumi.\arabic{equation}}
%\counterwithin{equation}{enumi}
%\renewcommand{\theequation}{\arabic{subsection}.\arabic{equation}}
\def\putbox#1#2#3{\makebox[0in][l]{\makebox[#1][l]{}\raisebox{\baselineskip}[0in][0in]{\raisebox{#2}[0in][0in]{#3}}}}
     \def\rightbox#1{\makebox[0in][r]{#1}}
     \def\centbox#1{\makebox[0in]{#1}}
     \def\topbox#1{\raisebox{-\baselineskip}[0in][0in]{#1}}
     \def\midbox#1{\raisebox{-0.5\baselineskip}[0in][0in]{#1}}
\vspace{3cm}
\title{Matrix Theory: Assignment 8}
\author{Debolena Basak\\ Roll No.: AI20RESCH11003\\ PhD Artificial Intelligence}

\maketitle
\newpage
%\tableofcontents
\bigskip
\renewcommand{\thefigure}{\theenumi}
\renewcommand{\thetable}{\theenumi}


\begin{abstract}
This document is based on orthonormal basis and orthonormal matrix.
\end{abstract}

%
Download all latex-tikz codes from 
%
\begin{lstlisting}
https://github.com/Debolena/EE5609/tree/master/Assignment_8
\end{lstlisting}
%
\section{Problem}
Let $\vec R^n, n\geq 2$ be equipped with standard inner product. Let $\vec {v_1, v_2,..., v_n}$ be n column vectors forming an orthornormal basis of $\vec R^n$. Let $\vec A$ be a n x n matrix formed by the column vectors, $\vec {v_1, v_2,..., v_n}$. Then, 
\begin{enumerate}
    \item $\vec A = \vec A^{-1}$
    \item $\vec A= \vec A^T$
    \item $\vec A^{-1}= \vec A^T $
    \item $Det\brak{\vec A} = 1 $
\end{enumerate}
\section{solution}
Given, $\vec {v_1, v_2,..., v_n}$ are orthonormal and form basis.\\
So, when they form column vectors of matrix $\vec A $, we can say that $\vec A$ is also orthonormal.
\begin{align}
    \therefore \quad & \vec {A^T A} =\vec I\\
    \implies & \vec{A^T A A^{-1}}= \vec {I A^{-1}}\\
    \implies & \vec {A^T} = \vec {A^{-1}}
\end{align}
Clearly, option 3 is the correct answer.
\subsection{Example:}
Let us consider an orthonormal basis for $\vec R^2$.\\[1em]
We can check that $\vec S$= \{ $\myvec{\frac{1}{\sqrt{5}}\\ \frac{2}{\sqrt{5}}}$, $\myvec{-\frac{2}{\sqrt{5}}\\ \frac{1}{\sqrt{5}}}$ \} forms an orthonormal basis.\\
Thus the matrix
\begin{align}
    \vec Q = \myvec{\frac{1}{\sqrt{5}} & -\frac{2}{\sqrt{5}}\\ \frac{2}{\sqrt{5}} & \frac{1}{\sqrt{5}} }
\end{align}
is the orthonormal matrix whose column vectors are the basis of $\vec R^2$.
\subsection{Option 4 cannot be true}
For an orthonormal matrix $\vec A$,
\begin{align}
    & \vec{A^TA} =\vec I\\
    \implies &\det \brak{\vec{A^TA}}=\det\brak{ \vec I}\\
    \implies & \det \brak{\vec A^T} \det\brak{\vec A}= 1\\
    \implies & {\det\brak{\vec A}}^2 = 1 \quad \because \det\brak{\vec A}= \det\brak{\vec{A^T}}\\
    \implies & \det\brak{\vec A} =\pm 1
\end{align}
Also, here a contradictory example:\\
Let,
\begin{align}
    \vec R = \myvec{-\frac{1}{\sqrt{5}} & -\frac{2}{\sqrt{5}}\\ -\frac{2}{\sqrt{5}} & \frac{1}{\sqrt{5}}}
\end{align}
Clearly, $\vec R$  is an orthonormal matrix and the column vectors of it form an orthonormal basis of $\vec R^2$.
But,
\begin{align}
    \det {\vec R} &= \mydet {-\frac{1}{\sqrt{5}} & -\frac{2}{\sqrt{5}}\\ -\frac{2}{\sqrt{5}} & \frac{1}{\sqrt{5}}}\\
    &= -1
\end{align}
From the above two arguments it is clear that option 4 cannot be true.
\end{document}
