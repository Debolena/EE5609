\documentclass[journal,12pt,twocolumn]{IEEEtran}
%
\usepackage{setspace}
\usepackage{gensymb}
%\doublespacing
\singlespacing

%\usepackage{graphicx}
%\usepackage{amssymb}
%\usepackage{relsize}
\usepackage[cmex10]{amsmath}
%\usepackage{amsthm}
%\interdisplaylinepenalty=2500
%\savesymbol{iint}
%\usepackage{txfonts}
%\restoresymbol{TXF}{iint}
%\usepackage{wasysym}
\usepackage{amsthm}
%\usepackage{iithtlc}
\usepackage{mathrsfs}
\usepackage{txfonts}
\usepackage{stfloats}
\usepackage{bm}
\usepackage{cite}
\usepackage{cases}
\usepackage{subfig}
%\usepackage{xtab}
\usepackage{longtable}
\usepackage{multirow}
%\usepackage{algorithm}
%\usepackage{algpseudocode}
\usepackage[utf8]{inputenc}
\usepackage{tikz}
\usetikzlibrary{positioning}
\usepackage{enumitem}
\usepackage{mathtools}
\usepackage{steinmetz}
\usepackage{tikz}
\usepackage{circuitikz}
\usepackage{verbatim}
\usepackage{tfrupee}
\usepackage[breaklinks=true]{hyperref}
%\usepackage{stmaryrd}
\usepackage{tkz-euclide} % loads  TikZ and tkz-base
%\usetkzobj{all}
\usetikzlibrary{calc,math}
\usepackage{listings}
    \usepackage{color}                                            %%
    \usepackage{array}                                            %%
    \usepackage{longtable}                                        %%
    \usepackage{calc}                                             %%
    \usepackage{multirow}                                         %%
    \usepackage{hhline}                                           %%
    \usepackage{ifthen}                                           %%
  %optionally (for landscape tables embedded in another document): %%
    \usepackage{lscape}     
\usepackage{multicol}
\usepackage{chngcntr}
%\usepackage{enumerate}

%\usepackage{wasysym}
%\newcounter{MYtempeqncnt}
\DeclareMathOperator*{\Res}{Res}
%\renewcommand{\baselinestretch}{2}
\renewcommand\thesection{\arabic{section}}
\renewcommand\thesubsection{\thesection.\arabic{subsection}}
\renewcommand\thesubsubsection{\thesubsection.\arabic{subsubsection}}

\renewcommand\thesectiondis{\arabic{section}}
\renewcommand\thesubsectiondis{\thesectiondis.\arabic{subsection}}
\renewcommand\thesubsubsectiondis{\thesubsectiondis.\arabic{subsubsection}}

% correct bad hyphenation here
\hyphenation{op-tical net-works semi-conduc-tor}
\def\inputGnumericTable{}                                 %%

\lstset{
%language=C,
frame=single, 
breaklines=true,
columns=fullflexible
}
%\lstset{
%language=tex,
%frame=single, 
%breaklines=true
%}

\begin{document}
%


\newtheorem{theorem}{Theorem}[section]
\newtheorem{problem}{Problem}
\newtheorem{proposition}{Proposition}[section]
\newtheorem{lemma}{Lemma}[section]
\newtheorem{corollary}[theorem]{Corollary}
\newtheorem{example}{Example}[section]
\newtheorem{definition}[problem]{Definition}
%\newtheorem{thm}{Theorem}[section] 
%\newtheorem{defn}[thm]{Definition}
%\newtheorem{algorithm}{Algorithm}[section]
%\newtheorem{cor}{Corollary}
\newcommand{\BEQA}{\begin{eqnarray}}
\newcommand{\EEQA}{\end{eqnarray}}
\newcommand{\define}{\stackrel{\triangle}{=}}
\bibliographystyle{IEEEtran}
%\bibliographystyle{ieeetr}
\providecommand{\mbf}{\mathbf}
\providecommand{\pr}[1]{\ensuremath{\Pr\left(#1\right)}}
\providecommand{\qfunc}[1]{\ensuremath{Q\left(#1\right)}}
\providecommand{\sbrak}[1]{\ensuremath{{}\left[#1\right]}}
\providecommand{\lsbrak}[1]{\ensuremath{{}\left[#1\right.}}
\providecommand{\rsbrak}[1]{\ensuremath{{}\left.#1\right]}}
\providecommand{\brak}[1]{\ensuremath{\left(#1\right)}}
\providecommand{\lbrak}[1]{\ensuremath{\left(#1\right.}}
\providecommand{\rbrak}[1]{\ensuremath{\left.#1\right)}}
\providecommand{\cbrak}[1]{\ensuremath{\left\{#1\right\}}}
\providecommand{\lcbrak}[1]{\ensuremath{\left\{#1\right.}}
\providecommand{\rcbrak}[1]{\ensuremath{\left.#1\right\}}}
\theoremstyle{remark}
\newtheorem{rem}{Remark}
\newcommand{\sgn}{\mathop{\mathrm{sgn}}}
\providecommand{\abs}[1]{\left\vert#1\right\vert}
\providecommand{\res}[1]{\Res\displaylimits_{#1}} 
\providecommand{\norm}[1]{\left\lVert#1\right\rVert}
%\providecommand{\norm}[1]{\lVert#1\rVert}
\providecommand{\mtx}[1]{\mathbf{#1}}
\providecommand{\mean}[1]{E\left[ #1 \right]}
\providecommand{\fourier}{\overset{\mathcal{F}}{ \rightleftharpoons}}
%\providecommand{\hilbert}{\overset{\mathcal{H}}{ \rightleftharpoons}}
\providecommand{\system}{\overset{\mathcal{H}}{ \longleftrightarrow}}
	%\newcommand{\solution}[2]{\textbf{Solution:}{#1}}
\newcommand{\solution}{\noindent \textbf{Solution: }}
\newcommand{\cosec}{\,\text{cosec}\,}
\providecommand{\dec}[2]{\ensuremath{\overset{#1}{\underset{#2}{\gtrless}}}}
\newcommand{\myvec}[1]{\ensuremath{\begin{pmatrix}#1\end{pmatrix}}}
\newcommand{\mydet}[1]{\ensuremath{\begin{vmatrix}#1\end{vmatrix}}}
%\numberwithin{equation}{section}
\numberwithin{equation}{subsection}
%\numberwithin{problem}{section}
%\numberwithin{definition}{section}
\makeatletter
\@addtoreset{figure}{problem}
\makeatother
\let\StandardTheFigure\thefigure
\let\vec\mathbf
%\renewcommand{\thefigure}{\theproblem.\arabic{figure}}
\renewcommand{\thefigure}{\theproblem}
%\setlist[enumerate,1]{before=\renewcommand\theequation{\theenumi.\arabic{equation}}
%\counterwithin{equation}{enumi}
%\renewcommand{\theequation}{\arabic{subsection}.\arabic{equation}}
\def\putbox#1#2#3{\makebox[0in][l]{\makebox[#1][l]{}\raisebox{\baselineskip}[0in][0in]{\raisebox{#2}[0in][0in]{#3}}}}
     \def\rightbox#1{\makebox[0in][r]{#1}}
     \def\centbox#1{\makebox[0in]{#1}}
     \def\topbox#1{\raisebox{-\baselineskip}[0in][0in]{#1}}
     \def\midbox#1{\raisebox{-0.5\baselineskip}[0in][0in]{#1}}
\vspace{3cm}
\title{Matrix Theory: Assignment 10}
\author{Debolena Basak\\ Roll No.: AI20RESCH11003\\ PhD Artificial Intelligence}

\maketitle
\newpage
%\tableofcontents
\bigskip
\renewcommand{\thefigure}{\theenumi}
\renewcommand{\thetable}{\theenumi}


\begin{abstract}
This document is based on checking some properties of orthogonal matrix.
\end{abstract}

%
Download all latex-tikz codes from 
%
\begin{lstlisting}
https://github.com/Debolena/EE5609/tree/master/Assignment_10
\end{lstlisting}
%
\section{Problem}
Let $\vec A$ be a real n x n orthogonal matrix, that is, $\vec{A^TA=AA^T=I_n}$, the n x n identity matrix. which of the following statements are necessarily true?
\begin{enumerate}
    \item $ \vec{<Ax,Ay> =  <x,y>}\quad \forall \vec{x,y}\in \vec R^n $
    \item All eigen values of $\vec A $ are either +1 or -1.
    \item The rows of $\vec A$ form an orthonormal basis of $\vec R^n.$
    \item $\vec A$ is diagonalizable over $\vec R.$ 
\end{enumerate}
\section{solution}
\subsection{Option 1}
\begin{align}
    \vec{<Ax,Ay>}&= \vec{\brak{Ax}^T Ay} \\
    &=\vec{ x^T A^T Ay}\\
    &= \vec{x^T y} \quad\because \vec{A^TA = I}\\
    &= \vec{<x,y>}
\end{align}
Hence, option 1 is correct.
\subsection{Option 2}
Let $\lambda$ be the eigen value and $\vec v$ be the eigen vector corresponding to it.\\
Then,
\begin{align}
    &\vec{Av} = \lambda\vec v\\
    \implies & \norm{\vec{Av}}^2 = \norm{\lambda\vec v}^2\\
    \implies & \norm{\vec{Av}}^2 = \abs{\lambda}^2 \norm{\vec v}^2 \label{eq:1}
\end{align}
Now,
\begin{align}
    \norm{\vec{Av}}^2 &= \brak{\vec{Av}}^T \vec{Av}\\
    &= \vec{v ^TA^TAv} \\
    &= \vec{v^TIv}\\
    &=\vec{v^Tv}\\
    &=\norm{\vec v}^2 \label{eq:2}
\end{align}
Comparing \eqref{eq:1} and \eqref{eq:2}, we get, 
\begin{align}
    \abs{\lambda}^2 =1 \\
    \implies\abs{\lambda}= \pm 1
\end{align}
But $\abs{\lambda}$ cannot be -1.
\begin{align}
    \therefore \abs{\lambda}=1\\
    \implies \lambda = \pm 1
\end{align}
Thus, option 2 is correct.
\subsection{Option 3}
Let $\vec{r_1, r_2, ..., r_n}$ denote the row vectors of $\vec A$.\\
Then, 
\begin{align}
    \vec{AA^T} = \myvec{\vec{r_1^Tr_1} & \vec{r_1^T r_2} & ... & \vec{r_1^Tr_n}\\ . & . &... & .\\ . & . &... & .\\. & . &... & .\\ \vec{r_n^Tr_1} & \vec{r_n^Tr_2} & ... & \vec{r_n^T r_n}} \label{eq:3}
\end{align}
But, $\vec A $ is orthogonal. So, $\vec {AA^T = I}$. It therefore follows that
\begin{enumerate}
    \item All diagonal elements of \eqref{eq:3} are 1.
    \item All off- diagonal elements of \eqref{eq:3} are 0.
\end{enumerate}
That is, for all $i, j =1, 2, ..., n $, 
\begin{align}
    \vec{r_i^T r_j} &= 1 ,\quad  i=j\\
    &= 0 , \quad i\neq j
\end{align}
Therefore, $\vec{r_1, r_2, ... r_n}$ are orthonormal and form a basis of $\vec R^n$.\\
Hence, option 3 is correct.
\subsection{Option 4}
Counter Example:\\
Let us consider a matrix  in $ \vec R^2 $
\begin{align}
    \vec Q= \myvec{ 0 & 1 \\-1 & 0}\\
    \therefore \vec Q^T =\myvec{ 0 & -1\\1 & 0}
\end{align}
Check that $\vec{AA^T= I},  \therefore \vec Q$ is orthogonal. \\
The characteristic equation is:
\begin{align}
    &\mydet{\vec Q- \lambda\vec I}=0\\
    \implies & \mydet{-\lambda & 1\\ -1 & -\lambda}=0\\
    \implies & \lambda^2 + 1 = 0\\
    \implies & \lambda= \pm i \notin \vec R
\end{align}
which implies $\vec Q$ is not diagonalizable over $\vec R$.\\

Hence, we can conclude that option 1, 2 and 3 are correct.
\end{document}
