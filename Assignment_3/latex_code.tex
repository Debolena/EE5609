\documentclass[journal,12pt,twocolumn]{IEEEtran}
%
\usepackage{setspace}
\usepackage{gensymb}
%\doublespacing
\singlespacing

%\usepackage{graphicx}
%\usepackage{amssymb}
%\usepackage{relsize}
\usepackage[cmex10]{amsmath}
%\usepackage{amsthm}
%\interdisplaylinepenalty=2500
%\savesymbol{iint}
%\usepackage{txfonts}
%\restoresymbol{TXF}{iint}
%\usepackage{wasysym}
\usepackage{amsthm}
%\usepackage{iithtlc}
\usepackage{mathrsfs}
\usepackage{txfonts}
\usepackage{stfloats}
\usepackage{bm}
\usepackage{cite}
\usepackage{cases}
\usepackage{subfig}
%\usepackage{xtab}
\usepackage{longtable}
\usepackage{multirow}
%\usepackage{algorithm}
%\usepackage{algpseudocode}
\usepackage[utf8]{inputenc}
\usepackage{tikz}
\usetikzlibrary{positioning}
\usepackage{enumitem}
\usepackage{mathtools}
\usepackage{steinmetz}
\usepackage{tikz}
\usepackage{circuitikz}
\usepackage{verbatim}
\usepackage{tfrupee}
\usepackage[breaklinks=true]{hyperref}
%\usepackage{stmaryrd}
\usepackage{tkz-euclide} % loads  TikZ and tkz-base
%\usetkzobj{all}
\usetikzlibrary{calc,math}
\usepackage{listings}
    \usepackage{color}                                            %%
    \usepackage{array}                                            %%
    \usepackage{longtable}                                        %%
    \usepackage{calc}                                             %%
    \usepackage{multirow}                                         %%
    \usepackage{hhline}                                           %%
    \usepackage{ifthen}                                           %%
  %optionally (for landscape tables embedded in another document): %%
    \usepackage{lscape}     
\usepackage{multicol}
\usepackage{chngcntr}
%\usepackage{enumerate}

%\usepackage{wasysym}
%\newcounter{MYtempeqncnt}
\DeclareMathOperator*{\Res}{Res}
%\renewcommand{\baselinestretch}{2}
\renewcommand\thesection{\arabic{section}}
\renewcommand\thesubsection{\thesection.\arabic{subsection}}
\renewcommand\thesubsubsection{\thesubsection.\arabic{subsubsection}}

\renewcommand\thesectiondis{\arabic{section}}
\renewcommand\thesubsectiondis{\thesectiondis.\arabic{subsection}}
\renewcommand\thesubsubsectiondis{\thesubsectiondis.\arabic{subsubsection}}

% correct bad hyphenation here
\hyphenation{op-tical net-works semi-conduc-tor}
\def\inputGnumericTable{}                                 %%

\lstset{
%language=C,
frame=single, 
breaklines=true,
columns=fullflexible
}
%\lstset{
%language=tex,
%frame=single, 
%breaklines=true
%}

\begin{document}
%


\newtheorem{theorem}{Theorem}[section]
\newtheorem{problem}{Problem}
\newtheorem{proposition}{Proposition}[section]
\newtheorem{lemma}{Lemma}[section]
\newtheorem{corollary}[theorem]{Corollary}
\newtheorem{example}{Example}[section]
\newtheorem{definition}[problem]{Definition}
%\newtheorem{thm}{Theorem}[section] 
%\newtheorem{defn}[thm]{Definition}
%\newtheorem{algorithm}{Algorithm}[section]
%\newtheorem{cor}{Corollary}
\newcommand{\BEQA}{\begin{eqnarray}}
\newcommand{\EEQA}{\end{eqnarray}}
\newcommand{\define}{\stackrel{\triangle}{=}}
\bibliographystyle{IEEEtran}
%\bibliographystyle{ieeetr}
\providecommand{\mbf}{\mathbf}
\providecommand{\pr}[1]{\ensuremath{\Pr\left(#1\right)}}
\providecommand{\qfunc}[1]{\ensuremath{Q\left(#1\right)}}
\providecommand{\sbrak}[1]{\ensuremath{{}\left[#1\right]}}
\providecommand{\lsbrak}[1]{\ensuremath{{}\left[#1\right.}}
\providecommand{\rsbrak}[1]{\ensuremath{{}\left.#1\right]}}
\providecommand{\brak}[1]{\ensuremath{\left(#1\right)}}
\providecommand{\lbrak}[1]{\ensuremath{\left(#1\right.}}
\providecommand{\rbrak}[1]{\ensuremath{\left.#1\right)}}
\providecommand{\cbrak}[1]{\ensuremath{\left\{#1\right\}}}
\providecommand{\lcbrak}[1]{\ensuremath{\left\{#1\right.}}
\providecommand{\rcbrak}[1]{\ensuremath{\left.#1\right\}}}
\theoremstyle{remark}
\newtheorem{rem}{Remark}
\newcommand{\sgn}{\mathop{\mathrm{sgn}}}
\providecommand{\abs}[1]{\left\vert#1\right\vert}
\providecommand{\res}[1]{\Res\displaylimits_{#1}} 
\providecommand{\norm}[1]{\left\lVert#1\right\rVert}
%\providecommand{\norm}[1]{\lVert#1\rVert}
\providecommand{\mtx}[1]{\mathbf{#1}}
\providecommand{\mean}[1]{E\left[ #1 \right]}
\providecommand{\fourier}{\overset{\mathcal{F}}{ \rightleftharpoons}}
%\providecommand{\hilbert}{\overset{\mathcal{H}}{ \rightleftharpoons}}
\providecommand{\system}{\overset{\mathcal{H}}{ \longleftrightarrow}}
	%\newcommand{\solution}[2]{\textbf{Solution:}{#1}}
\newcommand{\solution}{\noindent \textbf{Solution: }}
\newcommand{\cosec}{\,\text{cosec}\,}
\providecommand{\dec}[2]{\ensuremath{\overset{#1}{\underset{#2}{\gtrless}}}}
\newcommand{\myvec}[1]{\ensuremath{\begin{pmatrix}#1\end{pmatrix}}}
\newcommand{\mydet}[1]{\ensuremath{\begin{vmatrix}#1\end{vmatrix}}}
%\numberwithin{equation}{section}
\numberwithin{equation}{subsection}
%\numberwithin{problem}{section}
%\numberwithin{definition}{section}
\makeatletter
\@addtoreset{figure}{problem}
\makeatother
\let\StandardTheFigure\thefigure
\let\vec\mathbf
%\renewcommand{\thefigure}{\theproblem.\arabic{figure}}
\renewcommand{\thefigure}{\theproblem}
%\setlist[enumerate,1]{before=\renewcommand\theequation{\theenumi.\arabic{equation}}
%\counterwithin{equation}{enumi}
%\renewcommand{\theequation}{\arabic{subsection}.\arabic{equation}}
\def\putbox#1#2#3{\makebox[0in][l]{\makebox[#1][l]{}\raisebox{\baselineskip}[0in][0in]{\raisebox{#2}[0in][0in]{#3}}}}
     \def\rightbox#1{\makebox[0in][r]{#1}}
     \def\centbox#1{\makebox[0in]{#1}}
     \def\topbox#1{\raisebox{-\baselineskip}[0in][0in]{#1}}
     \def\midbox#1{\raisebox{-0.5\baselineskip}[0in][0in]{#1}}
\vspace{3cm}
\title{Matrix Theory: Assignment 3}
\author{Debolena Basak\\ Roll No.: AI20RESCH11003\\ PhD Artificial Intelligence}

\maketitle
\newpage
%\tableofcontents
\bigskip
\renewcommand{\thefigure}{\theenumi}
\renewcommand{\thetable}{\theenumi}
%\renewcommand{\theequation}{\theenumi}
%\begin{abstract}
%%\boldmath
%In this letter, an algorithm for evaluating the exact analytical bit error rate  (BER)  for the piecewise linear (PL) combiner for  multiple relays is presented. Previous results were available only for upto three relays. The algorithm is unique in the sense that  the actual mathematical expressions, that are prohibitively large, need not be explicitly obtained. The diversity gain due to multiple relays is shown through plots of the analytical BER, well supported by simulations. 
%
%\end{abstract}
% IEEEtran.cls defaults to using nonbold math in the Abstract.
% This preserves the distinction between vectors and scalars. However,
% if the journal you are submitting to favors bold math in the abstract,
% then you can use LaTeX's standard command \boldmath at the very start
% of the abstract to achieve this. Many IEEE journals frown on math
% in the abstract anyway.
% Note that keywords are not normally used for peerreview papers.
%\begin{IEEEkeywords}
%Cooperative diversity, decode and forward, piecewise linear
%\end{IEEEkeywords}
% For peer review papers, you can put extra information on the cover
% page as needed:
% \ifCLASSOPTIONpeerreview
% \begin{center} \bfseries EDICS Category: 3-BBND \end{center}
% \fi
%
% For peerreview papers, this IEEEtran command inserts a page break and
% creates the second title. It will be ignored for other modes.
%\IEEEpeerreviewmaketitle
\begin{abstract}
This a problem on congruency of triangles in a quadrilateral.
\end{abstract}

Download all latex-tikz codes from 
%
\begin{lstlisting}
https://github.com/Debolena/EE5609/blob/master/Assignment_3/latex_code.tex
\end{lstlisting}
%
\section{Problem}
ABCD is a quadrilateral in which AD = BC and $\angle{DAB}  = \angle{CBA}$ . Prove that
\begin{align}
& a)  \quad	\triangle ABD \cong \triangle BAC \\
& b)  \quad BD = AC \\ 
& c)  \quad \angle ABD = \angle BAC
\end{align}

\section{Figure}
\begin{figure}[!htb]
	\centering
    \centering
    \resizebox{\columnwidth}{!}{\input{quadrilateral.tex}}
	\caption{Quadrilateral ABCD}
\end{figure}

\section{Solution}
ABCD is a quadrilateral.
We are given that AD=BC and $\angle DAB= \angle CBA$.

We have to show that $\triangle ABD \cong \triangle BAC$.
\begin{align}
	& \angle DAB = \angle CBA & (Given) \\
	& AD = BC & (Given) \\
	& AB = BA & (Common \quad base)
\end{align}
Hence, by SAS congruency, $\triangle ABD \cong \triangle BAC.\\$

b) We have,
\begin{align}
 & \angle DAB  =  \angle CBA \\
 &\implies \cos\angle DAB  =  \cos\angle CBA \\
 & \label{eq1} \frac{(\vec A -\vec B)^T(\vec{A}-\vec{D})} {\norm{\vec{A}-\vec{B}} \norm{\vec{A}-\vec{D}} } 
 = \frac{(\vec B -\vec A)^T(\vec{B}-\vec{C})} {\norm{\vec{B}-\vec{A}} \norm{\vec{B}-\vec{C}}}
 \end{align}
,using the formula of dot product, i.e.,
\begin{align}
    &\vec{a}.\vec{b}=\norm{\vec{a}}. \norm{\vec{b}}. \cos\theta\\
    &\implies \cos\theta = \frac{\vec{a}.\vec{b}}{\norm{\vec{a}}.\norm{\vec{b}}}
\end{align}

We are given AD=BC and we know AB=BA always. Thus, 
\begin{align}
    &\norm{\vec{A}-\vec{D}}  =  \norm{\vec{B}-\vec{C}}\\
    &\norm{\vec{A}-\vec{B}}  =  \norm{\vec{B}-\vec{A}}
\end{align}
Then, from (\ref{eq1}), we have,
\begin{align}
& (\vec A -\vec B)^T(\vec{A}-\vec{D}) =  (\vec B -\vec A)^T(\vec{B}-\vec{C})\\
& \implies \norm{\vec{A}-\vec{B}}^2 - (\vec B -\vec D)^T(\vec{B}-\vec{A})  \nonumber\\
& \quad\quad=\norm{\vec{A}-\vec{B}}^2 - (\vec A -\vec C)^T(\vec{A}-\vec{B}) \\
& \implies (\vec B -\vec D)^T(\vec{B}-\vec{A}) = (\vec A -\vec C)^T(\vec{A}-\vec{B})\label{eq2}\\
& \implies \norm{\vec{B} - \vec{D}}\norm{\vec{B} - \vec{A}}\cos\angle ABD \nonumber \\
& \quad\quad= \norm{\vec{A} - \vec{C}}\norm{\vec{A} - \vec{B}}\cos\angle BAC \\
&\implies \norm{\vec{B} - \vec{D}}\cos\angle ABD  = \norm{\vec{A} - \vec{C}}\cos\angle BAC \label{eq3}
\end{align}

We have to prove: $\norm{\vec{B} - \vec{D}} = \norm{\vec{A} - \vec{C}}.\\$ 
From (\ref{eq2}),
\begin{align}
	& (\vec{B} -\vec{D})^T(\vec{B}-\vec{A}) = (\vec A -\vec C)^T(\vec{A}-\vec{B})
\end{align}
\begin{multline}
\implies \norm{\vec{B}-\vec{D}}^2 - (\vec D -\vec A)^T(\vec{D}-\vec{B}) \\
= \norm{\vec{A}-\vec{C}}^2 - (\vec C -\vec B)^T(\vec{C}-\vec{A})	
\end{multline}
\begin{multline}
\implies \norm{\vec{B}-\vec{D}}^2 - (\norm{\vec{A}-\vec{D}}^2 - (\vec A -\vec B)^T(\vec A -\vec D)) \\
 =\norm{\vec{A}-\vec{C}}^2 - (\norm{\vec{B}-\vec{C}}^2 - (\vec B -\vec A)^T(\vec B -\vec C))\\	
\end{multline}
We are given that,
\begin{align}
	\norm{\vec{A}-\vec{D}} =  \norm{\vec{B}-\vec{C}} 
\end{align}

\begin{multline}
\therefore \norm{\vec{B}-\vec{D}}^2 + (\vec A -\vec B)^T(\vec{A}-\vec{D}) = \\ \norm{\vec{A}-\vec{C}}^2 + (\vec B -\vec A)^T(\vec{B}-\vec{C}) 	
\end{multline}
\begin{multline}
\implies \norm{\vec{B}-\vec{D}}^2+ \norm{\vec{A}-\vec{B}}\norm{\vec{A}-\vec{D}}\cos \angle DAB = \\
 \norm{\vec{A}-\vec{C}}^2+\norm{\vec{B}-\vec{A}} \norm{\vec{B}-\vec{C}}\cos \angle CBA \label{eq4}	
\end{multline}

From the question, $\angle DAB = \angle CBA$ and $\norm{\vec{A}-\vec{D}} = \norm{\vec{B}-\vec{C}}$. We also know $\norm{\vec A -\vec B}= \norm{ \vec B- \vec A}$. 
Thus, from (\ref{eq4}), we get,
\begin{align}
	& \norm{\vec{B}-\vec{D}}^2 = \norm{\vec{A}-\vec{C}}^2 \\
	& \implies\label{eq5}\norm{\vec{B}-\vec{D}} = \norm{\vec{A}-\vec{C}}\\
	& \therefore BD=AC
\end{align}

c)  From (\ref{eq3}) and (\ref{eq5}), we have
\begin{align}
	& \cos\angle ABD = \cos\angle BAC \\
	& \implies\angle ABD = \angle BAC
\end{align}

\end{document}

